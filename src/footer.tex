\chapter{Postscript}
\label{postscript}

Congratulations: We've made it to the end!

\section{Looking Back}

We've covered quite a bit of ground so far. Here's a quick review...

\begin{outline}
\1 \Gls{functional programming}:
  \2 ``declarative'' programming style (recursion over immutable data
  structures, rather than looping over mutable arrays or pointer structures)
  \2 higher-order functions
  \2 polymorphism
\1 \textit{Logic}, the mathematical basis for software engineering:
  \[
    \frac{\text{logic}}{\text{software engineering}} \sim
    \frac{\text{calculus}}{\text{mechanical/civil engineering}}
  \]
  \2 inductively defined sets and relations
  \2 inductive proofs
  \2 proof objects
\1 \textit{Idris}, a dependently-typed programming language
  \2 functional core language
  \2 full \glspl{dependent type}
  \2 \gls{dependent pattern matching}
  \2 powerful \glspl{elaboration reflection}
\end{outline}


\section{Looking Forward}

If what you've seen so far has whetted your interest, you have two choices for
further reading in the \citefieldlink{sf}{title} series:

\begin{outline}
\1 \citefieldlink{plf}{title} (volume 2, by a set of authors similar to this
  book's) covers material that might be found in a graduate course on the theory
  of programming languages, including Hoare logic, operational semantics, and
  type systems.
\1 \citefieldlink{vfa}{title} (volume 3, by Andrew Appel) builds on the themes
  of functional programming and program verification in Coq, addressing a range
  of topics that might be found in a standard data structures course, with an
  eye to formal verification.
\end{outline}


\section{Other sources}

Here are some other good places to learn more...

\begin{outline}
\1 This book includes some optional chapters covering topics that you may
  find useful. Take a look at the table of contents and the chapter dependency
  diagram to find them.
\1 \todo[inline]{Mention Idris on StackOverflow}
\todo{Bibliography entries}
\1 Here are some great books on functional programming
  \2 Learn You a Haskell for Great Good, by Miran Lipovaca [Lipovaca
    2011].
  \2 Real World Haskell, by Bryan O'Sullivan, John Goerzen, and Don
  Stewart [O'Sullivan 2008]
  \2 ...and many other excellent books on Haskell, OCaml, Scheme, Racket,
  Scala, F$^\sharp$, etc., etc.
  \todo[inline]{And some further resources for Idris}
\1 If you're interested in real-world applications of formal verification to
  critical software, see the Postscript chapter of \citefieldlink{plf}{title}.
\1 For applications of Coq in building verified systems, the lectures and
  course materials for the 2017 DeepSpec Summer School are a great
  resource. \url{https://deepspec.org/event/dsss17/index.html}
\end{outline}

\phantomsection
\printglossaries

\phantomsection
\printbibliography
